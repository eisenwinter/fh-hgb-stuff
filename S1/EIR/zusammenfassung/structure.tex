\usepackage[utf8]{inputenc}  
\usepackage[ngerman]{babel} 
\usepackage{eulerpx}
\usepackage[T1]{fontenc}
\usepackage{textcomp}

\usepackage{fontspec}
\usepackage{polyglossia}
\setmainlanguage[babelshorthands=true]{german}
\usepackage[dvipsnames,table]{xcolor}


\usepackage[parfill]{parskip}


\usepackage{amsmath}
\usepackage{enumerate}
\usepackage[framemethod=tikz]{mdframed}

\usepackage{hyperref}

\usepackage{geometry}
\usepackage{float}
\usepackage{array}


\geometry{
	paper=a4paper, % Paper size, change to letterpaper for US letter size
	top=2.5cm, % Top margin
	bottom=3cm, % Bottom margin
	left=2.5cm, % Left margin
	right=2.5cm, % Right margin
	headheight=14pt, % Header height
	footskip=1.5cm, % Space from the bottom margin to the baseline of the footer
	headsep=1.2cm, % Space from the top margin to the baseline of the header
	%showframe, % Uncomment to show how the type block is set on the page
}


\mdfdefinestyle{question}{
	innertopmargin=1.2\baselineskip,
	innerbottommargin=0.8\baselineskip,
	roundcorner=0pt,
	nobreak,
	singleextra={%
		\draw(P-|O)node[xshift=1em,anchor=west,fill=white,draw,rounded corners=0pt]{%
		Beispiel \theQuestion\questionTitle};
	},
}

\newcounter{Question} % Stores the current question number that gets iterated with each new question

% Define a custom environment for numbered questions
\newenvironment{question}[1][\unskip]{
	\bigskip
	\stepcounter{Question}
	\newcommand{\questionTitle}{~#1}
	\begin{mdframed}[style=question]
}{
	\end{mdframed}
	\medskip
}


\mdfdefinestyle{answertyle}{%
linecolor=gray!50,linewidth=2pt,%
backgroundcolor=gray!10,
frametitlerule=true,%
frametitlebackgroundcolor=gray!20,
topline=false,bottomline=false,
innertopmargin=\topskip,
innerbottommargin=\topskip
}


\newenvironment{answer}[1][\unskip]{
	\bigskip
	\begin{mdframed}[style=answertyle]
		}{
			\end{mdframed}
			\medskip
}


\mdfdefinestyle{warning}{
	topline=false, bottomline=false,
	leftline=false, rightline=false,
	nobreak,
	singleextra={%
		\draw(P-|O)++(-0.5em,0)node(tmp1){};
		\draw(P-|O)++(0.5em,0)node(tmp2){};
		\fill[black,rotate around={45:(P-|O)}](tmp1)rectangle(tmp2);
		\node at(P-|O){\color{white}\scriptsize\bf !};
		\draw[very thick](P-|O)++(0,-1em)--(O);%--(O-|P);
	}
}

% Define a custom environment for warning text
\newenvironment{warn}[1][Warning:]{ % Set the default warning to "Warning:"
	\medskip
	\begin{mdframed}[style=warning]
		\noindent{\textbf{#1}}
}{
	\end{mdframed}
}

\mdfdefinestyle{info}{%
	topline=false, bottomline=false,
	leftline=false, rightline=false,
	nobreak,
	singleextra={%
		\fill[black](P-|O)circle[radius=0.4em];
		\node at(P-|O){\color{white}\scriptsize\bf i};
		\draw[very thick](P-|O)++(0,-0.8em)--(O);%--(O-|P);
	}
}

% Define a custom environment for information
\newenvironment{info}[1][Info:]{ % Set the default title to "Info:"
	\medskip
	\begin{mdframed}[style=info]
		\noindent{\textbf{#1}}
}{
	\end{mdframed}
}

\usepackage{tikz}
\usetikzlibrary{shadows,decorations.pathreplacing,angles,quotes,decorations,decorations.markings,decorations.text,calc,positioning,shapes,arrows,patterns}
\usepackage{tikzsymbols}
\usepackage{booktabs}

\pgfkeys{/pgf/decoration/.cd,
      distance/.initial=10pt
}  

\pgfdeclaredecoration{add dim}{final}{
\state{final}{% 
\pgfmathsetmacro{\dist}{5pt*\pgfkeysvalueof{/pgf/decoration/distance}/abs(\pgfkeysvalueof{/pgf/decoration/distance})} 
          %\pgfpathmoveto{\pgfpoint{0pt}{0pt}}             
          %\pgfpathlineto{\pgfpoint{0pt}{2*\dist}}   
          %\pgfpathmoveto{\pgfpoint{\pgfdecoratedpathlength}{0pt}} 
          %\pgfpathlineto{\pgfpoint{(\pgfdecoratedpathlength}{2*\dist}}
          %\pgfusepath{stroke} 
          \pgfsetdash{{0.1cm}{0.1cm}{0.1cm}{0.1cm}}{0cm}     
          \pgfsetarrowsstart{|}
          \pgfsetarrowsend{|}  
          \pgfpathmoveto{\pgfpoint{0pt}{\dist}}
          \pgfpathlineto{\pgfpoint{\pgfdecoratedpathlength}{\dist}} 
          \pgfusepath{stroke} 
          \pgfsetdash{}{0pt}
          \pgfpathmoveto{\pgfpoint{0pt}{0pt}}
          \pgfpathlineto{\pgfpoint{\pgfdecoratedpathlength}{0pt}}
}}

\tikzset{dim/.style args={#1,#2}{decoration={add dim,distance=#2},
                decorate,
                postaction={decorate,decoration={text along path,
                                                 raise=#2,
                                                 text align={align=center},
                                                 text={#1}}}}}

\def\spvec#1{\left(\vcenter{\halign{\hfil$##$\hfil\cr \spvecA#1;;}}\right)}
\def\spvecA#1;{\if;#1;\else #1\cr \expandafter \spvecA \fi}

\tikzset{
  grid/.style={step=1cm,gray!30,very thin},
  axis/.style={thick,<->},
  vect/.style={-stealth,line width=2pt},
  vnode/.style={midway,font=\scriptsize},
  proj/.style={dashed,color=gray!50,->},
  poly/.style={fill=blue!30,opacity=0.3}
}


% carry macro
\usepackage{mathtools}
\newcommand*{\carry}[1][1]{\overset{\textcolor{Gray}{#1}}}
\newcolumntype{B}[1]{r*{#1}{@{\,}r}}

% ieee drawing fun
\usepackage{bytefield}

% circuits
\usepackage{circuitikz}

% columns for circuits
\usepackage{multicol}


% circuit boxes

\mdfdefinestyle{cbox}{%
	topline=false, bottomline=false,
	leftline=false, rightline=false,
	nobreak,
	backgroundcolor= gray!5,
	singleextra={%
		\draw[thick](P-|O)++(0,0)--(O);%--(O-|P);
	}
}


\newenvironment{cbox}[1][\unskip]{ 
	\begin{mdframed}[style=cbox]
		\begin{center}
		\noindent{\small\textbf{#1}}
		
}{
	\end{center}
	\end{mdframed}
}


% kv diagramme 
\usepackage{kvmap}

% automaten

\usetikzlibrary{arrows,automata}